% \begin{figure*}
% \centering
%
\setlength{\unitlength}{.5\linewidth}



\usetikzlibrary{shapes}
\tikzset{
block/.style={draw,rectangle,
font=\normalsize,
% fill=red!30,
% minimum size=1.7cm,
% minimum height=1.7cm,
% minimum width=1.7cm,
minimum height=.2\unitlength,
minimum width=.2\unitlength,
text width=2cm,align=center},
block1/.style={draw,rectangle,
font=\normalsize,
% fill=red!30,
% minimum size=1.7cm,
% minimum height=1.7cm,
% minimum width=1.7cm,
minimum height=.2\unitlength,
minimum width=.4\unitlength,
text width=5cm,align=center},
oval/.style={draw,ellipse,fill=yellow!70},
decision/.style={draw,diamond,
aspect=2,
% fill=green!10,
% minimum size=1.7cm,
text width=4cm,
minimum height=.2\unitlength,
minimum width=.6\unitlength,
align=center,
inner sep=2pt}
}



\begin{adjustbox}{width=.7\textwidth}
%
\begin{tikzpicture}
\node[block] (X0) {$\bm{X}_0=[\bm{X}_1 | \bm{X}_2  | \bm{X}_3 ]$};

\node[block1, below=2*0.1\unitlength, fill=red!30] (X2) at (X0){$\bm{X}_{2}$};
\node[block, left=3*0.1\unitlength, fill=red!30] (X1) at (X2){$\bm{X}_{1}$};
\node[block, right=3*0.1\unitlength, fill=red!30] (X3) at (X2){$\bm{X}_{3}$};
% \node[block, right=3*0.1\unitlength, below=2*0.1\unitlength] (X3) at (X0){X3};




\node[block1, below=2*0.1\unitlength, fill=green!30] (Y2) at (X2){
% $\mbox{男性の平均寿命}=X_{2}\bm{\beta}+\bm{\varepsilon}$
$\mbox{平均寿命}=\bm{X}_{2}\bm{\beta}+\bm{\varepsilon}$
% $\mbox{男性の健康寿命}=\bm{X}_{2}\bm{\beta}+\bm{\varepsilon}$
$\mbox{健康寿命}=\bm{X}_{2}\bm{\beta}+\bm{\varepsilon}$
};
\node[block, left=3*0.1\unitlength, fill=green!30] (Y1) at (Y2){右と同様の作業};
\node[block, right=3*0.1\unitlength, fill=green!30] (Y3) at (Y2){左と同様の作業};



\node[block1, below=2*0.1\unitlength, fill=blue!30] (XX2) at (Y2){AIC変数選択};

\node[block, left=3*0.1\unitlength, fill=blue!30] (XX1) at (XX2){AIC変数選択};
\node[block, right=3*0.1\unitlength, fill=blue!30] (XX3) at (XX2){AIC変数選択};


\node[decision, below=2*0.1\unitlength]             (Z1) at (XX2){上記の説明変数を集め、47変数を超える場合は繰り返す};


\node[block, left=4*0.1\unitlength, above=1*0.1\unitlength] (side1) at (-6,-6){変数選択を繰り返し};
% \node[block, left=4*0.1\unitlength, above=1*0.1\unitlength] (side1) at (Y1){side1};


\node[block, below=3*0.1\unitlength]             (end) at (Z1){end\\
最終選択変数};



\node[block, above=3*0.1\unitlength]             (start) at (X0){start\\
説明変数行列 $
\bm{X}_{47\times p}
$から開始};


\draw[dashed, -latex] (X0) -- (X1);
\draw[-latex] (X0) -- (X2);
\draw[dashed, -latex] (X0) -- (X3);



\draw[dashed, -latex] (X1) -- (Y1);
\draw[-latex] (X2) -- (Y2);
\draw[dashed, -latex] (X3) -- (Y3);



\draw[dashed, -latex] (Y1) -- (XX1);
\draw[-latex] (Y2) -- (XX2);
\draw[dashed, -latex] (Y3) -- (XX3);


\draw[dashed] (XX1) -- (Z1);
\draw[-latex] (XX2) -- (Z1);
\draw[dashed] (XX3) -- (Z1);




\draw[-latex, line width=1mm]
(Z1)
% node[below=2*0.1\unitlength, left=.5*0.1\unitlength] {yes}
node[below=2] {Yes}
node[left=2] {No}
% node[above=.5*0.1\unitlength, left=2*0.1\unitlength] {no}
-|
(side1) ;


\draw[-latex, line width=1mm] (side1) |- (X0);


\draw[-latex, line width=1mm] (start) -- (X0);


\draw[-latex, line width=1mm] (Z1) -- (end);

\end{tikzpicture}
\end{adjustbox}
% \caption{変数選択アルゴリズム}
% \label{}
% \end{figure*}
