\documentclass{book}
\usepackage{otf}

\usepackage{mathptmx}
\usepackage{amsmath}





\begin{document}

% latex table generated in R 4.0.4 by xtable 1.8-4 package
% Sat Jun 26 12:33:48 2021
\begin{table}[ht]
\centering
\begingroup\tiny
\begin{tabular}{rrrrrl}
  \hline
 & Sepal.Length & Sepal.Width & Petal.Length & Petal.Width & Species \\ 
  \hline
1 & 5.10 & 3.50 & 1.40 & 0.20 & setosa \\ 
  2 & 4.90 & 3.00 & 1.40 & 0.20 & setosa \\ 
  3 & 4.70 & 3.20 & 1.30 & 0.20 & setosa \\ 
  4 & 4.60 & 3.10 & 1.50 & 0.20 & setosa \\ 
  5 & 5.00 & 3.60 & 1.40 & 0.20 & setosa \\ 
   \hline
\end{tabular}
\endgroup
\caption[bbb]{$\beta_0
  X_1+
  \beta_0 X_2$
  寿命} 
\label{tablelabel}
\end{table}


% latex table generated in R 4.0.4 by xtable 1.8-4 package
% Sun Jun 27 01:32:34 2021
\begin{table}[ht]
\centering
\begingroup\tiny
\begin{tabular}{rll}
  \hline
 & f\_var & m\_var \\ 
  \hline
1 & 受療率\_外来\_脳血管疾患\_2017 & 受療率\_入院\_心疾患\_2017 \\ 
  2 & 人口・世帯\_老年人口割合2020 & 自然環境\_年平均気温 \\ 
  3 & 人口・世帯\_生産年齢人口割合2020 & 健康・医療\_保健師数(人口10万人当たり) \\ 
  4 & 自然環境\_年平均気温 & 家計\_貯蓄現在高 \\ 
  5 & 労働\_完全失業率 & 人口・世帯\_高齢単身者世帯の割合 \\ 
  6 & 居住\_都市公園数(可住地面積100k㎡当たり) & 悪性新生物(大腸)\_年齢調整死亡率2015 \\ 
  7 & 高血圧疾患\_外来2014年 & 自己啓発・訓練−パソコンなどの情報処理 \\ 
  8 & 悪性新生物(大腸)\_年齢調整死亡率2015 & 一定のバリアフリー化率\_2018 \\ 
  9 & ボランティア総行動率−総数 & 自己啓発・訓練−芸術・文化 \\ 
  10 & 受療率\_外来\_心疾患\_2017 & 自己啓発・訓練−英語以外の外国語 \\ 
  11 & 居住\_一戸建住宅比率 &  \\ 
  12 & 75歳未満調整死亡率\_悪政新生物\_2019 &  \\ 
  13 & 診療所数\_2019 &  \\ 
  14 & バリアフリー\_手すりがある2018 &  \\ 
  15 & 循環器専門医数\_2020 &  \\ 
  16 & 家計\_スマートフォン所有数量(千世帯当たり) &  \\ 
  17 & ボランティア総行動率−高齢者を対象とした活動 &  \\ 
   \hline
\end{tabular}
\endgroup
\caption[bbb]{$\beta_0
  X_1+
  \beta_0 X_2$
  寿命} 
\label{tablelabel}
\end{table}



\ref{tablelabel}

\end{document}

