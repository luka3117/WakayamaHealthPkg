% latex table generated in R 4.0.4 by xtable 1.8-4 package
% Tue Jul 13 13:46:19 2021
\begin{table}[H]
\centering
\caption{共通変数(98個)}
\label{table_commom_d2.tex}
\begingroup\tiny

\begin{adjustbox}{width=1\textwidth}
\begin{tabular}{rll}
  \hline
 & var\_name\_Jpn...2 & var\_name\_Jpn...4 \\
  \hline
1 & 受療率\_入院\_悪性新生物\_2017 & 家計\_消費支出(一世帯当たり1か月) \\
  2 & 受療率\_入院\_心疾患\_2017 & 家計\_教育費割合(対消費支出) \\
  3 & 受療率\_入院\_脳血管疾患\_2017 & 家計\_教養娯楽費割合(対消費支出) \\
  4 & 受療率\_外来\_悪性新生物\_2017 & 家計\_貯蓄現在高 \\
  5 & 受療率\_外来\_心疾患\_2017 & 家計\_スマートフォン所有数量(千世帯当たり) \\
  6 & 受療率\_外来\_脳血管疾患\_2017 & 家計\_パソコン所有数量(千世帯当たり) \\
  7 & 病院数\_2019 & 家計\_自動車所有数量(千世帯当たり) \\
  8 & 診療所数\_2019 & 家計\_タブレット端末所有数量(千世帯当たり) \\
  9 & がん治療認定医数\_2020 & 人口・世帯\_高齢単身者世帯の割合 \\
  10 & 循環器専門医数\_2020 & 高血圧疾患\_入院2014年 \\
  11 & 内視鏡専門医数\_2020 & 高血圧疾患\_外来2014年 \\
  12 & 書籍購入代金\_2019 & 糖尿病\_入院2014年 \\
  13 & 人口・世帯\_年少人口割合2020 & 糖尿病\_外来2014年 \\
  14 & 人口・世帯\_老年人口割合2020 & 肉類\_2014 \\
  15 & 人口・世帯\_生産年齢人口割合2020 & 魚介類\_2014 \\
  16 & 人口・世帯\_粗死亡率2020 & 牛乳\_2014 \\
  17 & 人口・世帯\_共働き世帯割合2020 & 乳製品\_2014 \\
  18 & 自然環境\_年平均気温 & 卵\_2014 \\
  19 & 自然環境\_年平均相対湿度 & 大豆\_2014 \\
  20 & 自然環境\_降水量(年間) & 一定のバリアフリー化率\_2018 \\
  21 & 自然環境\_雪日数(年間) & 高度のバリアフリー化率\_2018 \\
  22 & 経済基盤\_県民所得 & バリアフリー\_手すりがある2018 \\
  23 & 行政基盤\_財政力指数 & バリアフリー\_廊下などが車いすで通行可能な幅2018 \\
  24 & 行政基盤\_収支比率 & バリアフリー\_段差のない屋内2018 \\
  25 & 行政基盤\_生活保護費割合(県財政) & 総実労働時間\_2016 \\
  26 & 行政基盤\_教育費割合(県財政) & 現金給与総額\_2016 \\
  27 & 教育\_最終学歴が大学・大学院卒の者の割合 & 生鮮肉(世帯数消費支出)\_2014 \\
  28 & 労働\_1次産業就業者比率 & 生鮮肉(世帯数消費支出)\_2015 \\
  29 & 労働\_2次産業就業者比率 & 生鮮肉(世帯数消費支出)\_2016 \\
  30 & 労働\_3次産業就業者比率 & 生鮮肉平均\_世帯数消費支出(2014〜2016) \\
  31 & 労働\_完全失業率 & 菓子類(世帯数消費支出)\_2014 \\
  32 & 文化・スポーツ\_図書館数(人口100万人当たり) & 菓子類(世帯数消費支出)\_2015 \\
  33 & 健康・医療\_一般診療所数(可住地面積100km$^2$当たり) & 菓子類(世帯数消費支出)\_2016 \\
  34 & 文化・スポーツ\_スポーツの行動者率 & 菓子類平均\_世帯数消費支出(2014〜2016) \\
  35 & 文化・スポーツ\_旅行・行楽行動者率 & 果物(世帯数消費支出)\_2014 \\
  36 & 居住\_持ち家比率 & 果物(世帯数消費支出)\_2015 \\
  37 & 居住\_一戸建住宅比率 & 果物(世帯数消費支出)\_2016 \\
  38 & 居住\_上水道給水人口比率 & 果物平均\_世帯数消費支出(2014〜2016) \\
  39 & 居住\_下水道普及比率 & 全国学力・学習状況(公立学校数)(中学校)\_2015 \\
  40 & 文化・スポーツ\_ボランティア活動行動者率 & 全国学力・学習状況(公立学校数)(小学生)\_2015 \\
  41 & 居住\_都市公園面積(人口1人当たり) & う蝕外来総数\_2014 \\
  42 & 居住\_都市公園数(可住地面積100km$^2$当たり) & 歯周疾患(歯肉炎)外来総数\_2014 \\
  43 & 健康・医療\_一般病院数(可住地面積100km$^2$当たり) & 骨の密度障害\_2014 \\
  44 & 居住\_主要道路舗装率 & 骨折\_2014 \\
  45 & 居住\_市町村舗装率 & 歯の補てつ\_2014 \\
  46 & 健康・医療\_一般歯科診療所数(人口10万人当たり) & アルツハイマー等(脳血管疾患)\_2014 \\
  47 & 健康・医療\_医療施設に従事する医師数(人口10万人当たり) & ジニ係数総世帯\_2014 \\
  48 & 健康・医療\_保健師数(人口10万人当たり) & 収入ジニ係数勤労世帯\_2014 \\
  49 & 安全\_交通事故発生件数(人口10万人当たり) &  \\
  50 & 家計\_実収入(一世帯当たり1か月) &  \\
   \hline
\end{tabular}
\end{adjustbox}

\endgroup
\end{table}
