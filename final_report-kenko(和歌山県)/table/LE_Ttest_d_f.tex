% latex table generated in R 4.0.4 by xtable 1.8-4 package
% Wed Jul 14 02:08:30 2021
\begin{table}[ht]
\centering
\caption{平均寿命が長い県と短い県のt-検定(女性)}
\label{LE_Ttest_d_f.tex}
\begingroup\tiny

\begin{adjustbox}{width=.5\textwidth}
\begin{tabular}{rlrrrrrrr}
  \hline
 & var\_name\_Jpn & estimate & estimate1 & estimate2 & statistic & p.value & conf.low & conf.high \\
  \hline
1 & 平均寿命\_2015 & -0.64 & 86.71 & 87.35 & -9.03 & 0.00 & -0.78 & -0.50 \\
  2 & 受療率\_外来\_脳血管疾患\_2017 & 15.62 & 87.67 & 72.04 & 1.79 & 0.08 & -2.00 & 33.25 \\
  3 & 人口・世帯\_老年人口割合2020 & 1.31 & 30.72 & 29.41 & 1.53 & 0.13 & -0.42 & 3.03 \\
  4 & 人口・世帯\_生産年齢人口割合2020 & -0.55 & 57.36 & 57.92 & -0.74 & 0.46 & -2.07 & 0.96 \\
  5 & 自然環境\_年平均気温 & -0.78 & 15.65 & 16.43 & -1.17 & 0.25 & -2.12 & 0.56 \\
  6 & 労働\_完全失業率 & 0.20 & 4.32 & 4.12 & 1.05 & 0.30 & -0.19 & 0.60 \\
  7 & 居住\_都市公園数(可住地面積100km$^2$当たり) & -32.76 & 93.90 & 126.66 & -0.87 & 0.39 & -108.52 & 43.01 \\
  8 & 高血圧疾患\_外来2014年 & 3.94 & 15.56 & 11.62 & 1.16 & 0.25 & -2.88 & 10.76 \\
  9 & 悪性新生物(大腸)\_年齢調整死亡率2015 & 1.15 & 12.34 & 11.18 & 2.94 & 0.01 & 0.36 & 1.95 \\
  10 & ボランティア総行動率−総数 & -1.96 & 26.84 & 28.80 & -2.34 & 0.02 & -3.65 & -0.27 \\
  11 & 受療率\_外来\_心疾患\_2017 & 17.52 & 123.00 & 105.48 & 2.25 & 0.03 & 1.77 & 33.28 \\
  12 & 居住\_一戸建住宅比率 & 3.70 & 66.22 & 62.52 & 1.06 & 0.30 & -3.35 & 10.75 \\
  13 & 75歳未満調整死亡率\_悪政新生物\_2019 & 3.35 & 57.08 & 53.73 & 2.71 & 0.01 & 0.86 & 5.85 \\
  14 & 診療所数\_2019 & -2.33 & 81.23 & 83.56 & -0.66 & 0.51 & -9.39 & 4.74 \\
  15 & バリアフリー\_手すりがある2018 & 54405.25 & 282179.17 & 227773.91 & 0.80 & 0.43 & -82333.25 & 191143.75 \\
  16 & 循環器専門医数\_2020 & -9.61 & 318.00 & 327.61 & -0.08 & 0.94 & -258.95 & 239.73 \\
  17 & 家計\_スマートフォン所有数量(千世帯当たり) & -25.21 & 1038.75 & 1063.96 & -0.76 & 0.45 & -92.43 & 42.01 \\
  18 & ボランティア総行動率−高齢者を対象とした活動 & -0.64 & 4.90 & 5.53 & -2.31 & 0.03 & -1.20 & -0.08 \\
   \hline
\end{tabular}
\end{adjustbox}

\endgroup
\end{table}
