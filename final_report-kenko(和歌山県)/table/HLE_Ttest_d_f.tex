% latex table generated in R 4.0.3 by xtable 1.8-4 package
% Wed Jul 14 22:16:37 2021
\begin{table}[ht]
\centering
\caption{健康寿命が長い県と短い県の平均値比較(女性)}
\label{HLE_Ttest_d_f.tex}
\begingroup\tiny

\begin{adjustbox}{width=.7\textwidth}
\begin{tabular}{rlrrr}
  \hline
 & var\_name\_Jpn & estimate & estimate1 & estimate2 \\
  \hline
1 & 健康寿命\_2016 & -1.05 & 74.43 & 75.48 \\
  2 & 受療率\_外来\_脳血管疾患\_2017 & 1.66 & 80.83 & 79.17 \\
  3 & 人口・世帯\_老年人口割合2020 & -0.22 & 29.97 & 30.19 \\
  4 & 人口・世帯\_生産年齢人口割合2020 & 0.56 & 57.91 & 57.35 \\
  5 & 自然環境\_年平均気温 & -0.47 & 15.80 & 16.27 \\
  6 & 労働\_完全失業率 & 0.29 & 4.36 & 4.07 \\
  7 & 居住\_都市公園数(可住地面積100km$^2$当たり) & 72.24 & 145.28 & 73.04 \\
  8 & 高血圧疾患\_外来2014年 & 6.12 & 16.62 & 10.50 \\
  9 & 悪性新生物(大腸)\_年齢調整死亡率2015 & 0.12 & 11.83 & 11.71 \\
  10 & ボランティア総行動率−総数 & -0.45 & 27.58 & 28.03 \\
  11 & 受療率\_外来\_心疾患\_2017 & 9.86 & 119.25 & 109.39 \\
  12 & 居住\_一戸建住宅比率 & -7.73 & 60.62 & 68.35 \\
  13 & 75歳未満調整死亡率\_悪政新生物\_2019 & 1.14 & 56.00 & 54.86 \\
  14 & 診療所数\_2019 & 7.13 & 85.86 & 78.73 \\
  15 & バリアフリー\_手すりがある2018 & 120460.69 & 314504.17 & 194043.48 \\
  16 & 循環器専門医数\_2020 & 233.39 & 436.92 & 203.52 \\
  17 & 家計\_スマートフォン所有数量(千世帯当たり) & 27.92 & 1064.75 & 1036.83 \\
  18 & ボランティア総行動率−高齢者を対象とした活動 & -0.03 & 5.19 & 5.23 \\
   \hline
\end{tabular}
\end{adjustbox}

\endgroup
\end{table}
