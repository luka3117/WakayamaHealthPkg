% !TEX root = ../report_kenko._wakayama_final.tex

%---------------------------------------------------------------------------
\chapter{本事業の概要、 はじめに}
%---------------------------------------------------------------------------
\section{背景}
日本は、 ここ25年の間、 平均寿命の延伸、 死亡率の低下により、 高齢化率が2016年において27\%を示しており、 既に「高齢社会(総人口に対して65歳以上の割合が14\%以上)」を過ぎて「超高齢社会」に入っている。\footnote{
	高齢化率とは総人口に対して65歳以上の高齢者人口が占める割合. 世界保健機構(WHO)や国連の定義によると、 高齢化率が7%を超えた社会を「高齢化社会」、 14%を超えた社会を「高齢社会」、 21%を超えた社会を「超高齢社会」と定義.
}
%また、 野村ら(Lancet、 2017)の研究では、 「疾病の地域格差(regional variation of disease)」問題を指摘し、
こうした現状を考慮すると
国・自治体の健康政策
も「健康の質」を上げる方向に
立案する必要性が求められる。
海外では既に国の保健対策をデータに基づいて行う変革が実施されており(Global Burden of Disease :  Generating Evidence、 Guiding Policy、 2010)、
和歌山県の保健活動にもデータに基づくエビデンスが必要と考えられる。



\section{目的}
和歌山県の健康・医療・介護に関するデータ、 経済状況・ボランティア参加率等の社会環境因子に関わるデータを利活用した現状分析を実施するとともに
和歌山県の位置づけや強み・弱みを把握し、
得られた新たな知見を県の施策に反映し、 県民の健康寿命の延伸を図る。
%健康の質を表す健康
寿命及び健康寿命を用いて統計解析し、
 今後、 和歌山県の健康及びヘルスケア産業における政策立案に役に立つ参考資料を示すことを目的とする。

\section{実施期間}
令和3年4月1日$\sim$令和3年7月31日まで


\section{データ}
データは和歌山県が収集した47都道府県の公的データ
%を活用し、
%公的データ
を活用することにする。
その他経済、 データの詳細は後述するが、
文化など多様なデータを用いる。


\section{方法}
%\tb{都道府県間の健康指標の比較を行った野村ら(Lancet、 2017)の研究の方法論を参考にしながら、 より和歌山県の視点から和歌山県を中心として統計分析を実施する。 }
データ分析には主に統計ソフトR 4.0.4
を利用した。
提供データの形式に適合する
% 最新の可視化
統計手法を取り入れ、
 % 探索データ解析を行うとともに、
平均寿命や健康寿命との関連を分析する。

影響を与える要因を探るため、
疾病と関連する医学的変数のほか、 社会的変数等を説明変数に取り入れ
分析を行う。
分析には、説明変数の変数選択、多変量解析よる次元縮約を行い、
分析可能なデータとして加工を加えた後に、
分析を行う。

%なお、 解析の方針と統計手法の詳細は、 担当者の間で意見交換しあい、 適切な方法を取り入れる。



\section{期待効果}

本県の現状に関して、
県民及び
県外から移住を検討する人に向けて
正しい情報発信の資料として活用されることが期待される。
ビックデータ時代に、 他県に先駆けて官学連携による健康データを活用する取り組みは、 データに基づく県政を推奨している国の方針とも当てはまるので、 他県のベンチマーク事例になることが期待される。




\chapter{データと変数}\label{chapter:data}

本研究で用いるデータは和歌山県の
「和歌山県データ利活用推進センター」が
2018年度
「健康寿命のためのデータ活用」
研究で
使用したデータの項目を参考し、
2021年時点で収集可能な同様の項目を収集して
滋賀大学にcsv形式
%(DataFormat.csv)
にて提供されたものである。


全てのデータは
インターネットから容易にダウンロードが可能である公的データである。
(データの出典は\tb{??}参考)。



データの次元は47都道府県を個体、
162項目を変数とする性別ごとの2つのデータセットである。
変数の中、平均寿命と健康寿命の2つの変数は分析の目的変数となり、
残りの160変数は説明変数として扱う。
本研究では
これらの2種類の寿命変数を
以下で寿命変数と呼ぶ。

160説明変数の項目は両者のデータセットで共通するが、
性別によってその数値が異なる変数がある。
例えば、
平均寿命は、和歌山県の男性の平均寿命は79.94歳、
和歌山県の男性の平均寿命は86.47歳のように性別ごとに異なる数値が得られる
性別ごとの情報が分かる変数であるが、
表\ref{table_commom_d2.tex}の「居住\_持ち家比率」変数は性別ごとに調べられ変数ではなく、男性のデータセットも女性のデータセットでも同じ数値が記入されている(例、和歌山県「持ち家比率」は性別に関係なく73%と同じ数値)。

%
% 説明変数の中には、性別ごとの情報が無い変数が含まれている。

%例えば、
%表\ref{table_mf_d.tex}の
%「平均寿命」は性別ごとの変数であり、
%和歌山県の男性の平均寿命は79.94歳、
%和歌山県の男性の平均寿命は86.47歳と性別ごとの情報が分かる変数であるが、

本研究では、平均寿命ように性別の区別のある変数を「性別変数」に、
「居住\_持ち家比率」のように性別の意味の無い変数(もしくは、データ収集時点で性別分けデータ
入手できなかった変数)を「共通変数」に
呼ぶことにする。
表\ref{table_mf_d2.tex}に本研究に用いた性別変数の一覧を、
表\ref{table_commom_d2.tex}に共通変数の一覧を示す。

%% latex table generated in R 4.0.3 by xtable 1.8-4 package
% Wed Jul 14 22:16:37 2021
\begin{table}[ht]
\centering
\caption{健康寿命が長い県と短い県の平均値比較(男性)} 
\label{HLE_Ttest_d_m.tex}
\begingroup\tiny
\begin{tabular}{rlrrr}
  \hline
 & var\_name\_Jpn & estimate & estimate1 & estimate2 \\ 
  \hline
1 & 健康寿命\_2016 & -0.85 & 71.65 & 72.50 \\ 
  2 & 受療率\_入院\_心疾患\_2017 & 9.12 & 62.25 & 53.13 \\ 
  3 & 自然環境\_年平均気温 & 0.17 & 16.11 & 15.94 \\ 
  4 & 健康・医療\_保健師数(人口10万人当たり) & 2.75 & 52.96 & 50.20 \\ 
  5 & 家計\_貯蓄現在高 & -1951.10 & 13542.38 & 15493.48 \\ 
  6 & 人口・世帯\_高齢単身者世帯の割合 & 1.72 & 12.20 & 10.48 \\ 
  7 & 悪性新生物(大腸)\_年齢調整死亡率2015 & 1.11 & 21.14 & 20.03 \\ 
  8 & 自己啓発・訓練−パソコンなどの情報処理 & -0.74 & 13.65 & 14.38 \\ 
  9 & 一定のバリアフリー化率\_2018 & -1.67 & 41.81 & 43.47 \\ 
  10 & 自己啓発・訓練−芸術・文化 & -0.46 & 8.37 & 8.82 \\ 
  11 & 自己啓発・訓練−英語以外の外国語 & -0.31 & 2.40 & 2.70 \\ 
   \hline
\end{tabular}
\endgroup
\end{table}
%<-送付用


% latex table generated in R 4.0.4 by xtable 1.8-4 package
% Tue Jul 13 13:46:19 2021
\begin{table}[H]
\centering
\caption{共通変数(98個)}
\label{table_commom_d2.tex}
\begingroup\tiny

\begin{adjustbox}{width=1\textwidth}
\begin{tabular}{rll}
  \hline
 & var\_name\_Jpn...2 & var\_name\_Jpn...4 \\
  \hline
1 & 受療率\_入院\_悪性新生物\_2017 & 家計\_消費支出(一世帯当たり1か月) \\
  2 & 受療率\_入院\_心疾患\_2017 & 家計\_教育費割合(対消費支出) \\
  3 & 受療率\_入院\_脳血管疾患\_2017 & 家計\_教養娯楽費割合(対消費支出) \\
  4 & 受療率\_外来\_悪性新生物\_2017 & 家計\_貯蓄現在高 \\
  5 & 受療率\_外来\_心疾患\_2017 & 家計\_スマートフォン所有数量(千世帯当たり) \\
  6 & 受療率\_外来\_脳血管疾患\_2017 & 家計\_パソコン所有数量(千世帯当たり) \\
  7 & 病院数\_2019 & 家計\_自動車所有数量(千世帯当たり) \\
  8 & 診療所数\_2019 & 家計\_タブレット端末所有数量(千世帯当たり) \\
  9 & がん治療認定医数\_2020 & 人口・世帯\_高齢単身者世帯の割合 \\
  10 & 循環器専門医数\_2020 & 高血圧疾患\_入院2014年 \\
  11 & 内視鏡専門医数\_2020 & 高血圧疾患\_外来2014年 \\
  12 & 書籍購入代金\_2019 & 糖尿病\_入院2014年 \\
  13 & 人口・世帯\_年少人口割合2020 & 糖尿病\_外来2014年 \\
  14 & 人口・世帯\_老年人口割合2020 & 肉類\_2014 \\
  15 & 人口・世帯\_生産年齢人口割合2020 & 魚介類\_2014 \\
  16 & 人口・世帯\_粗死亡率2020 & 牛乳\_2014 \\
  17 & 人口・世帯\_共働き世帯割合2020 & 乳製品\_2014 \\
  18 & 自然環境\_年平均気温 & 卵\_2014 \\
  19 & 自然環境\_年平均相対湿度 & 大豆\_2014 \\
  20 & 自然環境\_降水量(年間) & 一定のバリアフリー化率\_2018 \\
  21 & 自然環境\_雪日数(年間) & 高度のバリアフリー化率\_2018 \\
  22 & 経済基盤\_県民所得 & バリアフリー\_手すりがある2018 \\
  23 & 行政基盤\_財政力指数 & バリアフリー\_廊下などが車いすで通行可能な幅2018 \\
  24 & 行政基盤\_収支比率 & バリアフリー\_段差のない屋内2018 \\
  25 & 行政基盤\_生活保護費割合(県財政) & 総実労働時間\_2016 \\
  26 & 行政基盤\_教育費割合(県財政) & 現金給与総額\_2016 \\
  27 & 教育\_最終学歴が大学・大学院卒の者の割合 & 生鮮肉(世帯数消費支出)\_2014 \\
  28 & 労働\_1次産業就業者比率 & 生鮮肉(世帯数消費支出)\_2015 \\
  29 & 労働\_2次産業就業者比率 & 生鮮肉(世帯数消費支出)\_2016 \\
  30 & 労働\_3次産業就業者比率 & 生鮮肉平均\_世帯数消費支出(2014〜2016) \\
  31 & 労働\_完全失業率 & 菓子類(世帯数消費支出)\_2014 \\
  32 & 文化・スポーツ\_図書館数(人口100万人当たり) & 菓子類(世帯数消費支出)\_2015 \\
  33 & 健康・医療\_一般診療所数(可住地面積100km$^2$当たり) & 菓子類(世帯数消費支出)\_2016 \\
  34 & 文化・スポーツ\_スポーツの行動者率 & 菓子類平均\_世帯数消費支出(2014〜2016) \\
  35 & 文化・スポーツ\_旅行・行楽行動者率 & 果物(世帯数消費支出)\_2014 \\
  36 & 居住\_持ち家比率 & 果物(世帯数消費支出)\_2015 \\
  37 & 居住\_一戸建住宅比率 & 果物(世帯数消費支出)\_2016 \\
  38 & 居住\_上水道給水人口比率 & 果物平均\_世帯数消費支出(2014〜2016) \\
  39 & 居住\_下水道普及比率 & 全国学力・学習状況(公立学校数)(中学校)\_2015 \\
  40 & 文化・スポーツ\_ボランティア活動行動者率 & 全国学力・学習状況(公立学校数)(小学生)\_2015 \\
  41 & 居住\_都市公園面積(人口1人当たり) & う蝕外来総数\_2014 \\
  42 & 居住\_都市公園数(可住地面積100km$^2$当たり) & 歯周疾患(歯肉炎)外来総数\_2014 \\
  43 & 健康・医療\_一般病院数(可住地面積100km$^2$当たり) & 骨の密度障害\_2014 \\
  44 & 居住\_主要道路舗装率 & 骨折\_2014 \\
  45 & 居住\_市町村舗装率 & 歯の補てつ\_2014 \\
  46 & 健康・医療\_一般歯科診療所数(人口10万人当たり) & アルツハイマー等(脳血管疾患)\_2014 \\
  47 & 健康・医療\_医療施設に従事する医師数(人口10万人当たり) & ジニ係数総世帯\_2014 \\
  48 & 健康・医療\_保健師数(人口10万人当たり) & 収入ジニ係数勤労世帯\_2014 \\
  49 & 安全\_交通事故発生件数(人口10万人当たり) &  \\
  50 & 家計\_実収入(一世帯当たり1か月) &  \\
   \hline
\end{tabular}
\end{adjustbox}

\endgroup
\end{table}
%<-送付用
%
%\begin{landscape}
%% latex table generated in R 4.0.3 by xtable 1.8-4 package
% Mon Jul  5 22:05:32 2021
\begin{table}[ht]
\centering
\caption{共通変数(98個)}
\label{table_commom_d.tex}
\begingroup\tiny

\begin{adjustbox}{width=.5\textwidth}
\begin{tabular}{rllll}
  \hline
 & var\_name\_Jpn...2 & var\_name\_Jpn...4 & var\_name\_Jpn...6 & var\_name\_Jpn...8 \\
  \hline
1 & 受療率\_入院\_悪性新生物\_2017 & 行政基盤\_教育費割合(県財政) & 家計\_消費支出(一世帯当たり1か月) & 現金給与総額\_2016 \\
  2 & 受療率\_入院\_心疾患\_2017 & 教育\_最終学歴が大学・大学院卒の者の割合 & 家計\_教育費割合(対消費支出) & 生鮮肉(世帯数消費支出)\_2014 \\
  3 & 受療率\_入院\_脳血管疾患\_2017 & 労働\_1次産業就業者比率 & 家計\_教養娯楽費割合(対消費支出) & 生鮮肉(世帯数消費支出)\_2015 \\
  4 & 受療率\_外来\_悪性新生物\_2017 & 労働\_2次産業就業者比率 & 家計\_貯蓄現在高 & 生鮮肉(世帯数消費支出)\_2016 \\
  5 & 受療率\_外来\_心疾患\_2017 & 労働\_3次産業就業者比率 & 家計\_スマートフォン所有数量(千世帯当たり) & 生鮮肉平均\_世帯数消費支出(2014〜2016) \\
  6 & 受療率\_外来\_脳血管疾患\_2017 & 労働\_完全失業率 & 家計\_パソコン所有数量(千世帯当たり) & 菓子類(世帯数消費支出)\_2014 \\
  7 & 病院数\_2019 & 文化・スポーツ\_図書館数(人口100万人当たり) & 家計\_自動車所有数量(千世帯当たり) & 菓子類(世帯数消費支出)\_2015 \\
  8 & 診療所数\_2019 & 健康・医療\_一般診療所数(可住地面積100km$^2$当たり) & 家計\_タブレット端末所有数量(千世帯当たり) & 菓子類(世帯数消費支出)\_2016 \\
  9 & がん治療認定医数\_2020 & 文化・スポーツ\_スポーツの行動者率 & 人口・世帯\_高齢単身者世帯の割合 & 菓子類平均\_世帯数消費支出(2014〜2016) \\
  10 & 循環器専門医数\_2020 & 文化・スポーツ\_旅行・行楽行動者率 & 高血圧疾患\_入院2014年 & 果物(世帯数消費支出)\_2014 \\
  11 & 内視鏡専門医数\_2020 & 居住\_持ち家比率 & 高血圧疾患\_外来2014年 & 果物(世帯数消費支出)\_2015 \\
  12 & 書籍購入代金\_2019 & 居住\_一戸建住宅比率 & 糖尿病\_入院2014年 & 果物(世帯数消費支出)\_2016 \\
  13 & 人口・世帯\_年少人口割合2020 & 居住\_上水道給水人口比率 & 糖尿病\_外来2014年 & 果物平均\_世帯数消費支出(2014〜2016) \\
  14 & 人口・世帯\_老年人口割合2020 & 居住\_下水道普及比率 & 肉類\_2014 & 全国学力・学習状況(公立学校数)(中学校)\_2015 \\
  15 & 人口・世帯\_生産年齢人口割合2020 & 文化・スポーツ\_ボランティア活動行動者率 & 魚介類\_2014 & 全国学力・学習状況(公立学校数)(小学生)\_2015 \\
  16 & 人口・世帯\_粗死亡率2020 & 居住\_都市公園面積(人口1人当たり) & 牛乳\_2014 & う蝕外来総数\_2014 \\
  17 & 人口・世帯\_共働き世帯割合2020 & 居住\_都市公園数(可住地面積100km$^2$当たり) & 乳製品\_2014 & 歯周疾患(歯肉炎)外来総数\_2014 \\
  18 & 自然環境\_年平均気温 & 健康・医療\_一般病院数(可住地面積100km$^2$当たり) & 卵\_2014 & 骨の密度障害\_2014 \\
  19 & 自然環境\_年平均相対湿度 & 居住\_主要道路舗装率 & 大豆\_2014 & 骨折\_2014 \\
  20 & 自然環境\_降水量(年間) & 居住\_市町村舗装率 & 一定のバリアフリー化率\_2018 & 歯の補てつ\_2014 \\
  21 & 自然環境\_雪日数(年間) & 健康・医療\_一般歯科診療所数(人口10万人当たり) & 高度のバリアフリー化率\_2018 & アルツハイマー等(脳血管疾患)\_2014 \\
  22 & 経済基盤\_県民所得 & 健康・医療\_医療施設に従事する医師数(人口10万人当たり) & バリアフリー\_手すりがある2018 & ジニ係数総世帯\_2014 \\
  23 & 行政基盤\_財政力指数 & 健康・医療\_保健師数(人口10万人当たり) & バリアフリー\_廊下などが車いすで通行可能な幅2018 & 収入ジニ係数勤労世帯\_2014 \\
  24 & 行政基盤\_収支比率 & 安全\_交通事故発生件数(人口10万人当たり) & バリアフリー\_段差のない屋内2018 &  \\
  25 & 行政基盤\_生活保護費割合(県財政) & 家計\_実収入(一世帯当たり1か月) & 総実労働時間\_2016 &  \\
   \hline
\end{tabular}
\end{adjustbox}

\endgroup
\end{table}%<-送付用
%\end{landscape}


%
%\begin{landscape}
%% latex table generated in R 4.0.4 by xtable 1.8-4 package
% Tue Jul  6 12:48:04 2021
\begin{table}[ht]
\centering
\caption{性別変数(62個)}
\label{table_mf_d.tex}
\begingroup\tiny

\begin{adjustbox}{width=1.5\textwidth}
\begin{tabular}{rlll}
  \hline
 & var\_name\_Jpn...2 & var\_name\_Jpn...4 & var\_name\_Jpn...6 \\
  \hline
1 & 人口 & 趣味・娯楽−読書 & 喫煙率(計100本以上,6ヵ月以上\&直近1ヵ月)(40〜74歳)2014 \\
  2 & 75歳未満調整死亡率\_悪性新生物\_2018 & 自己啓発・訓練−学習・自己啓発・訓練率 & 20歳に比べて10kg体重増加(40〜74歳)2014 \\
  3 & 75歳未満調整死亡率\_悪政新生物\_2019 & 自己啓発・訓練−芸術・文化 & 歩く速度が速い(同年齢と比較)(40〜74歳)2014 \\
  4 & 年齢調整死亡率\_心疾患\_2015 & 自己啓発・訓練−英語 & 飲酒日1日当たり2合以上飲む割合(頻度)(40〜74歳)2014 \\
  5 & 年齢調整死亡率\_脳血管疾患\_2015 & 自己啓発・訓練−英語以外の外国語 & 毎日酒を飲む割合(頻度)(40〜74歳)2014 \\
  6 & 60歳以上人口\_2015 & 自己啓発・訓練−パソコンなどの情報処理 & 睡眠休養が十分とれている(40〜74歳)2014 \\
  7 & 学習率\_2016 & ボランティア活動−安全な生活のための活動 & 朝食を抜くことが週3回ある(40〜74歳)2014 \\
  8 & 読書率\_2016 & ボランティア活動−自然や環境の活動 & 夕食後に間食することが週3回ある(40〜74歳)2014 \\
  9 & スポーツ総行動率 & ボランティア活動−災害活動 & 野菜摂取量\_2016(20歳以上平均値(g/日) \\
  10 & スポーツ総行動率-器具を使ったトレーニング & 脳血管疾患\_年齢調整死亡率2015 & 食塩摂取量\_2016(20歳以上平均値(g/日) \\
  11 & スポーツ行動率−ウォーキング & 悪性新生物(胃)\_年齢調整死亡率2015 & BMI平均値\_2016(男性20〜69歳)(女性40〜69歳)(単位Kg/m$^2$) \\
  12 & 旅行・行楽−旅行・行楽・観光総行動率 & 悪性新生物(大腸)\_年齢調整死亡率2015 & 歩数\_2016(20歳以上平均値(歩/日) \\
  13 & 旅行・行楽−旅行率 & 悪性新生物(肝及び肝内胆管)\_年齢調整死亡率2015 &  \\
  14 & 旅行・行楽−行楽率 & 悪性新生物(気管、気管支及び肺)\_年齢調整死亡率2015 &  \\
  15 & 旅行・行楽−観光率 & 悪性新生物(乳房)\_年齢調整死亡率2015 &  \\
  16 & ボランティア総行動率−総数 & 悪性新生物(子宮)\_年齢調整死亡率2015 &  \\
  17 & ボランティア総行動率−まちづくり活動 & 心疾患\_年齢調整死亡率2015 &  \\
  18 & ボランティア総行動率−国際協力活動 & 肺炎\_年齢調整死亡率2015 &  \\
  19 & ボランティア総行動率−健康や医療サービスに関係した活動 & 急性心筋梗塞\_年齢調整死亡率2015 &  \\
  20 & ボランティア総行動率−高齢者を対象とした活動 & 血圧を下げる薬の使用(40〜64歳)2014 &  \\
  21 & ボランティア総行動率−障害者を対象とした活動 & インシュリン注射、血糖を下げる薬の使用(40〜74歳)2014 &  \\
  22 & ボランティア総行動率−子供を対象とした活動 & コレステロールを下げる薬の使用(40〜74歳)2014 &  \\
  23 & 趣味・娯楽−趣味娯楽総行動率 & 就寝前の2時間以内に夕食(40〜74歳)2014 &  \\
  24 & 趣味・娯楽−園芸・庭いじり・ガーデニング & 日常生活において歩行等の身体活動(1日1時間以上実施)(40〜74歳)2014 &  \\
  25 & 趣味・娯楽−スポーツ観覧 & 軽く汗をかく運動週2回(40〜74歳)2014 &  \\
   \hline
\end{tabular}
\end{adjustbox}

\endgroup
\end{table}
%<-送付用
%\end{landscape}



\input{table/table_mf_d2.tex}%<-送付用






% 種類と考えられる。
%
%
% 64変数は県の情報を表す。
%
%
% は98変数が、
%
% 性別ごとの意味があるか無いかによって

%
% 変数によっては性別に関係なく同様の数値が記載されている。
%
% 県単位で記載された変数が混在しており、



%データの扱いには入力ミスなどが起こらないように、
%十分に注意を払い、
%また、 滋賀大と和歌山県それぞれ、
%同様のデータについて、
%相互チェックを行なった.
%その後、
%複数回に渡り
%データに関する
%初回のデータの項目に幾つかの変更がなされ、
%本研究に用いられるデータとして絞られた.
%
%以下に、 変更点の一部をを記す.
%\begin{itemize}
%	\item
%		熊本県の欠損値について、 国の示す推定方法により数値を記載 -
%熊本県女性の健康寿命の推定値に置き換える。

%\item
%野菜摂取量\_2016(20歳以上平均値(g/日)
%食塩摂取量\_2016(20歳以上平均値(g/日)
%BMI平均値\_2016(男性20〜69歳)(女性40〜69歳)(単位Kg/u)
%歩数\_2016(20歳以上平均値(歩/日)等の
%データの収集元URLを修正.
