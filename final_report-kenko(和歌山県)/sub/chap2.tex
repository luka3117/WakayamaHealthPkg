% !TEX root = ../report_kenko._wakayama_final.tex



% # ========================
\chapter{方法}\label{chapter:methods}
% # ========================


第\ref{chapter:data}章では
今回の分析に用いるデータとその変数の詳細について紹介した。

この章で
分析の方向性と基本設定を紹介するとともに、
分析の全体像を示す。
第\ref{chapter:data}章で紹介したデータをどのように分析に用いて、
分析をおこない、最終的
第???章で結論づけた健康に影響を与える因子がどのように導きだしたかについて説明する。

本研究ではモデルの基本設定として
性別ごとの寿命変数が目的変数となり、
残り160変数は説明変数の候補群となる設定である。
したがって、
説明変数として用いられる変数を目的変数である寿命変数の
原因になる可能性を内在していると仮定した上図\ref{ModelStructure}に示すようなモデルを
考慮するのが自然的な発想である。


\begin{figure}[H]
	% \includegraphics{/path/to/figure}
%
%\documentclass[dvipdfmx]{jsarticle}
%%\documentclass[border=0.125cm]{standalone}
%\usepackage{tikz}
%\usetikzlibrary{positioning}
%\begin{document}

\begin{center}
\begin{adjustbox}{width=.5\textwidth}
\tikzset{%
  every neuron/.style={
    rectangle,
    draw,
    minimum size=1cm,
    minimum width=3cm
  },
  neuron missing/.style={
    draw=none,
    scale=4,
    text height=0.333cm,
    execute at begin node=\color{black}$\vdots$
  },
}


%\begin{center}


\begin{tikzpicture}[x=2cm, y=1.5cm, >=stealth]

% \foreach \m/\l [count=\y] in {1,2,3,missing,4} % <-original
\foreach \m/\l [count=\y] in {1,2,3,4}
  \node [every neuron/.try, neuron \m/.try] (input-\m) at (0,2.5-\y) {};

\foreach \m [count=\y] in {1}
  \node [every neuron/.try, neuron \m/.try ] (hidden-\m) at (3,2-\y*2) {};

%
% % node for error :: by jc lee
% \foreach \l [count=\i] in {1,2,3,p}
%   \draw [->] (input-\i) -- ++(-1,0)
%     node [above, left] {$\varepsilon_\l$};

% node for x1...xp :: by jc lee
% to above in the circle use : \node [above]
\foreach \l [count=\i] in {女性の平均,男性の平均,女性の健康,男性の健康}
  \node [] at (input-\i) {\l 寿命};

% node for F1,F2 :: by jc lee
\foreach \l [count=\i] in {1}
  \node [] at (hidden-\i) {$X_1, \cdots, X_{162}$};
%

\foreach \i in {1,2,3,4}
  \foreach \j in {1}
    \draw [<-] (input-\i) -- (hidden-\j);

\foreach \l [count=\x from 0] in {目的変数}
  \node [align=center, above] at (\x*2,2) {\l \\ 結果};

\foreach \l [count=\x from 1] in {説明変数}
  \node [align=center, above] at (1+\x*2,2) {\l \\ 原因};


  % \foreach \l [count=\i] in {1,n}
  %   \draw [->] (output-\i) -- ++(1,0)
  %     node [above, midway] {$O_\l$};

  % \foreach \i in {1,...,4}
  %   \foreach \j in {1,...,2}
  %     \draw [<-] (input-\i) -- (hidden-\j);
  %
  % \foreach \i in {2}
  %   \foreach \j in {2}
  %     \draw [<-] (input-\i) -- (hidden-\j);


  % \foreach \i in {1,...,2}
  %   \foreach \j in {1,...,2}
  %     \draw [->] (hidden-\i) -- (output-\j);

% === following is original comment add by jc lee===
% \foreach \l [count=\x from 0] in {観測変数, 共通因子}
%   \node [align=center, above] at (\x*2,2) {\l \\ 層};
% === following is original comment add by jc lee end ===

\end{tikzpicture}
%\end{center}
%\end{document}
\end{adjustbox}
\end{center}
%<-送付用
	\caption{モデリングの基本構造}
	\label{ModelStructure}
\end{figure}



しかし、本研究で用いられるデータは都道府県がデータの47個体となるデータであるため、
図\ref{ModelStructure}のように全ての説明変数を一つのモデルに取り入れて
160個の回帰係数
$\beta_0 ,\beta_0 ,\cdots , \beta_{160}$
を推定することが技術的にできない。

以上の理由で、モデリングを進むための前処理として、
160変数の中から
比較的に扱いやすい数の説明変数の数を減らす必要がある。
この前処理は2段階で構成される。

1段階の前処理では160変数の中から一定の基準(基準については後述)を満たす変数をを選択し, 説明変数の数を減らす。
結果として、男性の場合は、10個の変数が、
女性の場合は、17個の変数がをそれぞれの性別の説明変数として選択された(\ref{chapter:VarSelection}章参照)。
2段階の前処理では1段階の前処理で選択された変数に対して、
多変量分析の因子分析を用いて2つの共通因子を抽出する(第\ref{chapter:FA}章)。
本研究ではこの2つの共通因子を説明変数とし,寿命変数を目的変数に設定して
第\ref{glm}章以降でモデリングを行う。

言い換えると、以上の前処理が終わった時点で、
モデルへの適用が困難であた
図\ref{ModelStructure}の構造は、
多様なモデルが適用できるような単純な構造となり(図\ref{ModelSuppression})、
目的変数(寿命変数)に影響を与える変数を
2つの共通因子を通して調べることができる。

\begin{figure}[H]
	% \includegraphics{/path/to/figure}
%
%\documentclass[dvipdfmx]{jsarticle}
%%\documentclass[border=0.125cm]{standalone}
%\usepackage{tikz}
%\usetikzlibrary{positioning}
%\begin{document}

\begin{center}
\begin{adjustbox}{width=.5\textwidth}
\tikzset{%
  every neuron/.style={
    rectangle,
    draw,
    minimum size=1cm,
    minimum width=5cm
  },
  neuron missing/.style={
    draw=none,
    scale=4,
    text height=0.333cm,
    execute at begin node=\color{black}$\vdots$
  },
}


%\begin{center}


\begin{tikzpicture}[x=2cm, y=1.5cm, >=stealth]

% \foreach \m/\l [count=\y] in {1,2,3,missing,4} % <-original
\foreach \m/\l [count=\y] in {1,2,3,4}
  \node [every neuron/.try, neuron \m/.try] (input-\m) at (0,2.5-\y) {};

\foreach \m [count=\y] in {1}
  \node [every neuron/.try, neuron \m/.try ] (hidden-\m) at (3,2-\y*2) {};

%
% % node for error :: by jc lee
% \foreach \l [count=\i] in {1,2,3,p}
%   \draw [->] (input-\i) -- ++(-1,0)
%     node [above, left] {$\varepsilon_\l$};

% node for x1...xp :: by jc lee
% to above in the circle use : \node [above]
\foreach \l [count=\i] in {女性の平均,男性の平均,女性の健康,男性の健康}
  \node [] at (input-\i) {\l 寿命};

% node for F1,F2 :: by jc lee
\foreach \l [count=\i] in {1}
  \node [] at (hidden-\i) {$\mbox{共通因子1}_1, \mbox{共通因子2}_2$};
%

\foreach \i in {1,2,3,4}
  \foreach \j in {1}
    \draw [<-] (input-\i) -- (hidden-\j);

\foreach \l [count=\x from 0] in {目的変数}
  \node [align=center, above] at (\x*2,2) {\l \\ 結果};

\foreach \l [count=\x from 1] in {説明変数}
  \node [align=center, above] at (1+\x*2,2) {\l \\ 原因};


  % \foreach \l [count=\i] in {1,n}
  %   \draw [->] (output-\i) -- ++(1,0)
  %     node [above, midway] {$O_\l$};

  % \foreach \i in {1,...,4}
  %   \foreach \j in {1,...,2}
  %     \draw [<-] (input-\i) -- (hidden-\j);
  %
  % \foreach \i in {2}
  %   \foreach \j in {2}
  %     \draw [<-] (input-\i) -- (hidden-\j);


  % \foreach \i in {1,...,2}
  %   \foreach \j in {1,...,2}
  %     \draw [->] (hidden-\i) -- (output-\j);

% === following is original comment add by jc lee===
% \foreach \l [count=\x from 0] in {観測変数, 共通因子}
%   \node [align=center, above] at (\x*2,2) {\l \\ 層};
% === following is original comment add by jc lee end ===

\end{tikzpicture}
%\end{center}
%\end{document}
\end{adjustbox}
\end{center}
%<-送付用
	\caption{説明変数の共通因子を用いたモデリング}
	\label{ModelSuppression}
\end{figure}



例えば、男性の平均寿命$Y$とすると線形回帰モデルは
\begin{eqnarray}
E(Y|F_1, F_2)=\beta_0 +\beta_0F_1 + \beta_2 F_2
\end{eqnarray}
に表される。
右辺の係数$\beta_1, \beta_2, $は共通因子が
左辺の平均寿命の期待値へ与える
影響と考えればよい。


第\ref{glm}章と
\ref{chapter:bayes}章では2つの共通因子を用いて様々なモデリングを適用し、
寿命への影響を分析する。
第\ref{glm}章では
線形回帰モデルを含むより一般化した一般化線形モデルを利用し、寿命変数と2つの共通因子との関連性に有無を調べる。

最後に\ref{chapter:bayes}章ではベイズモデルを用い、
2つの共通因子が寿命変数の事後分布にどのような影響を与えるかを分析する。


主にR version 4.0.4を用いて行った。
また、説明変数は標準化処理を行った。


% 一般化線形モデルとベイズモデルを使い、次元縮約変数が性別ごとに寿命変数に
% どのような影響をあたえるかについて議論する。



%
%
% \begin{eqnarray}
% 	E(Y|X_1=x_1, X_2=x_2, \cdots, X_p=x_p)=\\
% 	f(x_1, x_2, \cdots, x_p)
% \end{eqnarray}



%
% となる。
%
%
% 性別ごとに説明変数のデータの次元は47都道府県を個体、
% 変数は162変数となる。
%
% \begin{eqnarray}
% y=\beta_0 +\beta_0 +\beta_0 +
% \end{eqnarray}
%
%
% モデルに用いるには数が多い。
% したがって、説明変数のモデルに用いられる
% 変数の数をは分析の目的変数となり、
% 残りの160変数は説明変数として扱う。
%
% 目的変数の中、平均寿命は、和歌山県の男性の平均寿命は79.94歳
%
%
%  変数の説明
% 「詳は添付資料」
