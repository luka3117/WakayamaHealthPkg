%
%\documentclass[dvipdfmx]{jsarticle}
%%\documentclass[border=0.125cm]{standalone}
%\usepackage{tikz}
%\usetikzlibrary{positioning}
%\begin{document}

\begin{center}
\begin{adjustbox}{width=.5\textwidth}
\tikzset{%
  every neuron/.style={
    rectangle,
    draw,
    minimum size=1cm,
    minimum width=3cm
  },
  neuron missing/.style={
    draw=none,
    scale=4,
    text height=0.333cm,
    execute at begin node=\color{black}$\vdots$
  },
}


%\begin{center}


\begin{tikzpicture}[x=2cm, y=1.5cm, >=stealth]

% \foreach \m/\l [count=\y] in {1,2,3,missing,4} % <-original
\foreach \m/\l [count=\y] in {1,2,3,4}
  \node [every neuron/.try, neuron \m/.try] (input-\m) at (0,2.5-\y) {};

\foreach \m [count=\y] in {1}
  \node [every neuron/.try, neuron \m/.try ] (hidden-\m) at (3,2-\y*2) {};

%
% % node for error :: by jc lee
% \foreach \l [count=\i] in {1,2,3,p}
%   \draw [->] (input-\i) -- ++(-1,0)
%     node [above, left] {$\varepsilon_\l$};

% node for x1...xp :: by jc lee
% to above in the circle use : \node [above]
\foreach \l [count=\i] in {女性の平均,男性の平均,女性の健康,男性の健康}
  \node [] at (input-\i) {\l 寿命};

% node for F1,F2 :: by jc lee
\foreach \l [count=\i] in {1}
  \node [] at (hidden-\i) {$X_1, \cdots, X_{160}$};
%

\foreach \i in {1,2,3,4}
  \foreach \j in {1}
    \draw [<-] (input-\i) -- (hidden-\j);

\foreach \l [count=\x from 0] in {目的変数}
  \node [align=center, above] at (\x*2,2) {\l \\ 結果};

\foreach \l [count=\x from 1] in {説明変数}
  \node [align=center, above] at (1+\x*2,2) {\l \\ 原因};


  % \foreach \l [count=\i] in {1,n}
  %   \draw [->] (output-\i) -- ++(1,0)
  %     node [above, midway] {$O_\l$};

  % \foreach \i in {1,...,4}
  %   \foreach \j in {1,...,2}
  %     \draw [<-] (input-\i) -- (hidden-\j);
  %
  % \foreach \i in {2}
  %   \foreach \j in {2}
  %     \draw [<-] (input-\i) -- (hidden-\j);


  % \foreach \i in {1,...,2}
  %   \foreach \j in {1,...,2}
  %     \draw [->] (hidden-\i) -- (output-\j);

% === following is original comment add by jc lee===
% \foreach \l [count=\x from 0] in {観測変数, 共通因子}
%   \node [align=center, above] at (\x*2,2) {\l \\ 層};
% === following is original comment add by jc lee end ===

\end{tikzpicture}
%\end{center}
%\end{document}
\end{adjustbox}
\end{center}
